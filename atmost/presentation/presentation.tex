\documentclass[compress,table,xcolor=table]{beamer}
\mode<presentation>
\usetheme{Montpellier}
\hypersetup{pdfstartview={Fit}}

\usecolortheme{beaver}
\useoutertheme{smoothbars}

\setbeamertemplate{bibliography item}[text]
\addtobeamertemplate{frametitle}{}{\vspace{-1em}}



\usepackage[utf8]{inputenc}
\usepackage{url}
\usepackage{caption}
\usepackage{enumitem}
\usepackage{dirtytalk}
\usepackage{multirow}
\usepackage{graphicx}

\newcommand{\fitfig}[1]{%
  \makebox[\linewidth][c]{%
    \begin{minipage}{\dimexpr\textwidth+1.2in\relax}
    \includegraphics[
        width=\textwidth,
        height=\textheight,
        keepaspectratio
    ]{/tmp/atmost_presentation/#1}
    \end{minipage}%
  }%
}

\definecolor{light-gray}{gray}{0.95}

\newcommand{\shelltext}[1]{\texttt{\colorbox{light-gray}{#1}}}

\usepackage{listings}
\lstset{
    basicstyle=\footnotesize\ttfamily,
    language=C++,
    breaklines=true,
    tabsize=2,
    escapeinside={(*@}{@*)}
}


\setlist[itemize,1]{label=$\bullet$}
\setlist[itemize,2]{label=$\bullet$}
\setlist[itemize,3]{label=$\bullet$}

\begin{document}
% ------------------------------------------------------------------------------
\title{External process synchronization\\for fun and profit}
% - Intro ----------------------------------------------------------------------
\section{Intro}
\frame{\titlepage}
% ------------------------------------------------------------------------------
\begin{frame}
  \frametitle{The LLVM project}
  \begin{itemize}
      \item{\LARGE{\say{a collection of modular and reusable compiler and toolchain technologies}}}
        \begin{itemize}
          \item{\shelltext{llvm} -- code generator and optimizer for many CPUs}
          \item{\shelltext{lldb} -- debugger}
          \item{\shelltext{lld} -- linker}
          \item{\shelltext{clang} -- C/C++/Objective-C compiler}
          \item{\shelltext{clang-tidy} -- C/C++/Objective-C static analysis tool}
          \item{\shelltext{clang-format} -- C/C++/Objective-C code format tool}
          \item{\ldots}
        \end{itemize}
      \item{\url{https://llvm.org/}}
      \item{\url{https://github.com/llvm/llvm-project}}
  \end{itemize}
\end{frame}
% - Motivation -----------------------------------------------------------------
\section{Motivation}
\subsection{Building bigger projects}
% ------------------------------------------------------------------------------
\begin{frame}
  \frametitle{Let's build \shelltext{llvm} and \shelltext{clang}}
  \begin{itemize}
      \large
      \item[$\$\rangle$]{\shelltext{git clone \url{https://github.com/llvm/llvm-project.git}}}
      \item[$\$\rangle$]{\shelltext{mkdir \_build}}
      \item[$\$\rangle$]{\shelltext{cd \_build}}
      \item[$\$\rangle$]{\shelltext{cmake -DCMAKE\_BUILD\_TYPE=Debug ..}}
      \item[$\$\rangle$]{\shelltext{make}}
  \end{itemize}
\end{frame}
% ------------------------------------------------------------------------------
\begin{frame}
  \frametitle{\shelltext{make}: system resource usage}
  \fitfig{resusage-llvm-16GB-ld-noccache-1j.pdf}
\end{frame}
% ------------------------------------------------------------------------------
\begin{frame}
  \frametitle{That didn't go so well\ldots}
  \begin{itemize}
    \Large
    \item{System resource usage is low}
    \item{Build takes cca. 5 hours -- unacceptably long}
    \item{Let's try some tricks:}
    \begin{itemize}
      \large
      \item use \shelltext{ccache}
      \item use more \shelltext{make} jobs
      \item use \shelltext{distcc} for distributed compilation
    \end{itemize}
  \end{itemize}
\end{frame}
% ------------------------------------------------------------------------------
\begin{frame}
  \frametitle{ccache -- compiler cache}
  \begin{itemize}
      \item{\LARGE{\say{speeds up recompilation by caching previous compilations and detecting when the same compilation is being done again}}}
      \item{\url{https://ccache.dev/}}
  \end{itemize}
\end{frame}
% ------------------------------------------------------------------------------
\begin{frame}
  \frametitle{\shelltext{make}: uncached clean build vs. 100\% cached}
  \fitfig{time-llvm-16GB-ld-noccache-vs-ccache-1j.pdf}
\end{frame}
% ------------------------------------------------------------------------------
\begin{frame}
  \frametitle{More jobs: \shelltext{make clean \&\& make -j 2}}
  \fitfig{resusage-llvm-16GB-ld-noccache-2j.pdf}
\end{frame}
% ------------------------------------------------------------------------------
\begin{frame}
  \frametitle{That's somewhat better\ldots}
  \begin{itemize}
    \Large
    \item{System resource usage is still low}
    \item{Build takes more than 2 and half hours -- still too long}
    \item{Parallelization shows potential}
  \end{itemize}
\end{frame}
% ------------------------------------------------------------------------------
\begin{frame}
  \frametitle{Keep going: \shelltext{make clean \&\&  make -j 3}}
  \fitfig{resusage-llvm-16GB-ld-noccache-3j.pdf}
\end{frame}
% ------------------------------------------------------------------------------
\begin{frame}
  \frametitle{What happened?}
  \begin{itemize}
    \Large
    \item{System usage starts to climb}
    \begin{itemize}
      \large
      \item{especially memory usage}
      \item{signs of correlation with linker execution}
    \end{itemize}
    \item{Build \emph{failed} after 2 hours 20 minutes -- looooong}
    \begin{itemize}
      \large
      \item{Linux OOM\footnote{Out Of Memory} process killer, kills some linker jobs}
    \end{itemize}
    \item{Parallelization still shows potential}
  \end{itemize}
\end{frame}
% ------------------------------------------------------------------------------
\begin{frame}
  \frametitle{Meh. MOAR jobs! \shelltext{make clean \&\&  make -j 7}}
  \fitfig{resusage-llvm-16GB-ld-noccache-7j.pdf}
\end{frame}
% ------------------------------------------------------------------------------
\begin{frame}
  \frametitle{Not great, not terrible}
  \begin{itemize}
    \Large
    \item{System usage very high toward the end of the build}
    \begin{itemize}
      \large
      \item{especially memory usage}
      \item{still correlated with linker execution}
    \end{itemize}
    \item{Build \emph{failed} after 1 hours 5 minutes -- getting better}
    \begin{itemize}
      \large
      \item{OOM process killer, kills some linker jobs}
    \end{itemize}
    \item{Parallelization rules!}
  \end{itemize}
\end{frame}
% ------------------------------------------------------------------------------
\begin{frame}
  \frametitle{Still come cores left\dots \shelltext{make clean \&\&  make -j 9}}
  \fitfig{reboot-llvm-16GB-ld-noccache-9j.pdf}
\end{frame}
% ------------------------------------------------------------------------------
\begin{frame}
  \frametitle{What have we learned so far}
  \begin{itemize}
    \LARGE
    \item{Most of the build time is spent by compilation}
    \item{Compilation caching -- good.}
    \item{Compilation paralelization -- good.}
    \begin{itemize}
      \Large
    \item{We have not even {\LARGE really used} \shelltext{distcc} yet.}
    \end{itemize}
    \item{Linking paralelization -- bad! Why?}
    \begin{itemize}
      \Large
      \item{Let's have a closer look!}
    \end{itemize}
  \end{itemize}
\end{frame}
% ------------------------------------------------------------------------------
\begin{frame}
  \frametitle{Linker memory usage vs link time -- no cache}
    \fitfig{scatter-llvm-buildtime-ramusage-noccache-8G-8C.pdf}
\end{frame}
% ------------------------------------------------------------------------------
\begin{frame}
  \frametitle{Linker memory usage vs link time -- cached}
    \fitfig{scatter-llvm-buildtime-ramusage-ccache-8G-8C.pdf}
\end{frame}
% ------------------------------------------------------------------------------
\begin{frame}
  \frametitle{Link target memory requirements -- classification}
    \fitfig{hist-llvm-link-mem-req.pdf}
\end{frame}
% ------------------------------------------------------------------------------
\begin{frame}
    \frametitle{Linking \shelltext{llvm} and \shelltext{clang} in a nutshell}
  \begin{itemize}
    \large
    \item{Around 100 different link targets}
    \begin{itemize}
      \normalsize
      \item{Executables, Shared libraries}
      \item{Many of them are {\normalsize small}.}
      \item{Few of the {\LARGE huge}.}
    \end{itemize}
    \item{Most of the linking is done {\Large towards the end} of the build process.}
    \begin{itemize}
      \large
      \item[$\implies$]{Using many parellel jobs, multiple instances of linker run concurrently.}
      \begin{itemize}
        \item[$\implies$]{Many {\Large big targets} are linked {\Large at the same time}.}
        \begin{itemize}
          \item[$\implies$]{\Huge{OOM!}}
        \end{itemize}
      \end{itemize}
    \end{itemize}
    \item{BTW: so much swapping is bad for the SSDs.}
  \end{itemize}
\end{frame}
% ------------------------------------------------------------------------------
\begin{frame}
    \frametitle{Making parallel linking work}
  \begin{itemize}
    \LARGE
    \item Let's try some additional tricks:
    \begin{itemize}
      \Large
      \item Use GNU \shelltext{gold} instead of GNU \shelltext{ld}.
      \begin{itemize}
        \large
        \item[$+$] Faster than \shelltext{ld}.
        \item[$+$] {\Large Uses less memory} than \shelltext{ld}.
	    \normalsize 
		\item[$-$] {Can link only in ELF format.}
      \end{itemize}
      \item Use \shelltext{zram}.
      \begin{itemize}
        \normalsize
        \item a Linux kernel module for creating a compressed block device in RAM
        \item can be used as swap device -- part of swap is in compressed RAM
      \end{itemize}
      \item Make only some targets
      \begin{itemize}
        \normalsize
        \item Practical when {\large developing and debugging}.
        \item Sometimes you just need to do \shelltext{make all}.
      \end{itemize}
    \end{itemize}
  \end{itemize}
\end{frame}
% ------------------------------------------------------------------------------
\begin{frame}
  \frametitle{\shelltext{zram}+\shelltext{distcc}+\shelltext{ccache}+\shelltext{gold}+\shelltext{make -j 12}}
  \fitfig{resusage-llvm-16GB-gold-ccache-12j.pdf}
\end{frame}
% ------------------------------------------------------------------------------
\begin{frame}
    \frametitle{If only\ldots}
  \begin{itemize}
    \LARGE
    \item \ldots we could
    \begin{itemize}
      \Large
      \item prevent so many big targets from being linked at once,
      \item prevent excessive swapping to disk,
    \end{itemize}
    \item but still
    \begin{itemize}
      \Large
      \item use the hardware resources efficiently,
	  \item use available remote cores via \shelltext{distcc} to full extent.
    \end{itemize}
  \end{itemize}
\end{frame}
% - Solution -------------------------------------------------------------------
\section{Solution}
\subsection{The idea}
% ------------------------------------------------------------------------------
\begin{frame}
  \frametitle{Let's synchronize the execution of \shelltext{ld}/\shelltext{gold}}
  \begin{itemize}
    \LARGE
    \item \shelltext{atmost} FTW!
    \begin{itemize}
	  \large
      \item Simple utility written in C.
      \item Takes a command-line (executable + arguments).
      \item Waits until specified conditions are met.
	  \item Executes the command line.
	  \item {\small\url{https://github.com/matus-chochlik/various}/atmost}
	  
    \end{itemize}
  \end{itemize}
\end{frame}
% ------------------------------------------------------------------------------
\begin{frame}[fragile]
  \frametitle{\shelltext{atmost --help}}
\scriptsize
\begin{verbatim}
ATMOST(1)                 General Commands Manual                ATMOST(1)

NAME
       atmost  - tool for limiting the concurrent execution of a specified
       executable(s).

SYNOPSIS
       atmost [-v|--verbose] [-f|--file file-path]
              [-s|--socket socket-path] [-r|--reset]
              [-i|--sleep-interval seconds] [-c|--print-current]
              [-C|--print-all-current] [-n|--max-instances count]
              [-l|--max-cpu-load-1m percent]
              [-L|--max-cpu-load-5m percent] [-m|--min-avail-ram percent]
              [-M|--min-free-ram percent] [-S|--min-free-swap percent]
              [-p|--max-total-procs count] [-tc|--max-cpu-temp temp]
              [-tg|--max-gpu-temp temp] [-tb|--max-bat-temp temp]
              [-io|--max-io-ops count] [-nw|--max-nw-speed speed]
              [-snw|--slow-nw-speed speed] [-- executable [args...]]

       atmost --help
\end{verbatim}
\end{frame}
% ------------------------------------------------------------------------------
\begin{frame}
  \frametitle{\shelltext{atmost} -- description}
  \Large
    \say{The atmost utility allows to~{\LARGE limit concurrent execution}
    of~a~single executable
    or~a~set of~executables, depending on~various criteria like CPU load,
    memory and~swap usage, thermal zone temperatures, current number
    of~I/O~operations, etc., or just {\LARGE by a maximum number}.}
\end{frame}
% ------------------------------------------------------------------------------
\begin{frame}
  \frametitle{\shelltext{atmost -n NUMBER -- EXECUTABLE} -- basic usage}
  \begin{itemize}
    \Large
    \item Limits the concurrent execution of \shelltext{EXECUTABLE} to the specified \shelltext{NUMBER} of instances.
    \item Uses an IPC semaphore set\footnote{see \shelltext{man 2 semget}}.
    \begin{itemize}
      \large 
      \item {\em Acquires\footnote{see \shelltext{man 2 semop}}} the semaphore before executing \shelltext{EXECUTABLE}.
      \item {\em Releases} the semaphore after executing \shelltext{EXECUTABLE}.
      \item \shelltext{realpath}\footnote{see \shelltext{man 3 realpath}} of \shelltext{EXECUTABLE} is used as the semaphore set token\footnote{see \shelltext{man 3 ftok}}.
    \end{itemize}
  \end{itemize}
\end{frame}
% ------------------------------------------------------------------------------
\begin{frame}[fragile]
  \frametitle{Using \shelltext{atmost} -- wrapper scripts}
  \begin{itemize}
    \large
    \item In order to synchronize a \shelltext{EXECUTABLE} with \shelltext{atmost}:
    \begin{itemize}
      \item Create a shell script with the name of \shelltext{EXECUTABLE}.
      \item From the script call \shelltext{atmost} with appropriate arguments.
      \item Put the wrapper script to \shelltext{\$\{PATH\}}
    \end{itemize}
  \end{itemize}
  \shelltext{\$ vim /opt/bin/ld}
  \begin{verbatim}
  #!/bin/bash
  atmost -n 2 -- /usr/bin/x86_64-linux-gnu-gold "${@}"
  \end{verbatim}
  \shelltext{\$ chmod u+x /opt/bin/ld}
  \shelltext{\$ export PATH="/opt/bin:\$\{PATH\}"}\\
\end{frame}
% ------------------------------------------------------------------------------
\begin{frame}
  \frametitle{\shelltext{atmost -n 2 -- ld}}
  \begin{itemize}
  \Large
  \item [+] Empirically we have found that 2 instances of \shelltext{ld} can run
    safely with 16GB RAM.
  \item [+] Simple
  \item [+] Low overhead
  \item [-] Too coarse and restrictive
    \begin{itemize}
      \large
      \item The majority of the \shelltext{llvm} targets has low RAM requirements for linking.
      \item Targets from other projects also have low linker RAM requirements.
      \item We could safely use {\Large more parallel link jobs}.
    \end{itemize}
  \end{itemize}
\end{frame}
% ------------------------------------------------------------------------------
\begin{frame}
  \frametitle{What if\dots}
  \begin{itemize}
    \large
  \item \dots we could {\LARGE determine} if it is safe to start the linker {\LARGE per each invocation}?
    \begin{itemize}
      \LARGE
      \item Enter \shelltext{atmost\_driver}!
      \begin{itemize}
        \large
        \item Running as a local \shelltext{AF\_UNIX} socket server.
        \item \shelltext{atmost} is the client (with the \shelltext{--socket PATH} option).
        \item When \shelltext{atmost} is started, it connects to the {\em driver} and sends information about the wrapped \shelltext{EXECUTABLE}.
        \item When the {\em driver} determines that it's safe to run \shelltext{EXECUTABLE}, it responds to \shelltext{atmost}.
        \item Then \shelltext{atmost} executes \shelltext{EXECUTABLE}.
      \end{itemize}
    \end{itemize}
  \end{itemize}
\end{frame}
% ------------------------------------------------------------------------------
\begin{frame}
  \frametitle{Let's make it even more flexible}
  \begin{itemize}
    \LARGE
    \item \shelltext{atmost\_driver} callbacks:
    \begin{itemize}
      \large
      \item The \shelltext{atmost\_driver} implements the reusable {\Large common
        server functionality}.
      \item The specific logic determining when it is safe to start executing,
        is implemented in a {\Large set of callbacks}.
      \item \shelltext{load\_callback\_data}
      \item \shelltext{save\_callback\_data}
      \item \shelltext{process\_initialized}
      \item \shelltext{let\_process\_go}
      \item \shelltext{process\_finished}
    \end{itemize}
  \end{itemize}
\end{frame}
% ------------------------------------------------------------------------------
\begin{frame}
  \frametitle{\shelltext{atmost}+\shelltext{atmost\_driver}+callbacks}
  \fitfig{diagram-components.pdf}
\end{frame}
% ------------------------------------------------------------------------------
\end{document}

