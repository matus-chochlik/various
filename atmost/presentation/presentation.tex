\documentclass[compress,table,xcolor=table]{beamer}
\mode<presentation>
\usetheme{Montpellier}
\hypersetup{pdfstartview={Fit}}

\usecolortheme{beaver}
\useoutertheme{smoothbars}

\setbeamertemplate{bibliography item}[text]
\addtobeamertemplate{frametitle}{}{\vspace{-1em}}

\usepackage[utf8]{inputenc}
\usepackage{url}
\usepackage{gensymb}
\usepackage{caption}
\usepackage{enumitem}
\usepackage{dirtytalk}
\usepackage{multirow}
\usepackage{graphicx}

\newcommand{\fitfig}[1]{%
  \makebox[\linewidth][c]{%
    \begin{minipage}{\dimexpr\textwidth+1.2in\relax}
    \includegraphics[
        width=\textwidth,
        height=\textheight,
        keepaspectratio
    ]{/tmp/atmost_presentation/#1}
    \end{minipage}%
  }%
}

\definecolor{light-gray}{gray}{0.95}

\newcommand{\shelltext}[1]{\texttt{\colorbox{light-gray}{#1}}}

\usepackage{listings}
\lstset{
    basicstyle=\normalsize\ttfamily,
    backgroundcolor = \color{light-gray},
    language=Python,
    breaklines=true,
    showstringspaces=false,
    keywordstyle=\itshape\bfseries\color{red!65!black},
    tabsize=2,
    escapeinside={(*@}{@*)}
}


\setlist[itemize,1]{label=$\bullet$}
\setlist[itemize,2]{label=$\bullet$}
\setlist[itemize,3]{label=$\bullet$}

\begin{document}
% ------------------------------------------------------------------------------
\title{External process synchronization\\for fun and profit}
% - Intro ----------------------------------------------------------------------
\section{Intro}
\frame{\titlepage}
% ------------------------------------------------------------------------------
\begin{frame}
  \frametitle{The LLVM project}
  \begin{itemize}
      \item{\LARGE{\say{a collection of modular and reusable compiler and toolchain technologies}}}
        \begin{itemize}
          \item{\shelltext{llvm} -- code generator and optimizer for many CPUs}
          \item{\shelltext{lldb} -- debugger}
          \item{\shelltext{lld} -- linker}
          \item{\shelltext{clang} -- C/C++/Objective-C compiler}
          \item{\shelltext{clang-tidy} -- C/C++/Objective-C static analysis tool}
          \item{\shelltext{clang-format} -- C/C++/Objective-C code format tool}
          \item{\ldots}
        \end{itemize}
      \item{\url{https://llvm.org/}}
      \item{\url{https://github.com/llvm/llvm-project}}
  \end{itemize}
\end{frame}
% - Motivation -----------------------------------------------------------------
\section{Motivation}
% ------------------------------------------------------------------------------
\begin{frame}
  \frametitle{Let's build \shelltext{llvm} and \shelltext{clang}}
  {\shelltext{\$ git clone \url{https://github.com/llvm/llvm-project.git}}}\\
  {\shelltext{\$ mkdir \_build}}\\
  {\shelltext{\$ cd \_build}}\\
  {\shelltext{\$ cmake -DCMAKE\_BUILD\_TYPE=Debug ..}}\\
  {\shelltext{\$ make}}\\
\end{frame}
% ------------------------------------------------------------------------------
\begin{frame}
  \frametitle{\shelltext{make}: system resource usage}
  \fitfig{resusage-llvm-16GB-ld-noccache-1j.pdf}
\end{frame}
% ------------------------------------------------------------------------------
\begin{frame}
  \frametitle{That didn't go so well\ldots}
  \begin{itemize}
    \Large
    \item{System resource usage is low}
    \item{Build takes cca. 5 hours -- unacceptably long}
    \item{Let's try some tricks:}
    \begin{itemize}
      \large
      \item use \shelltext{ccache}
      \item use more \shelltext{make} jobs
      \item use \shelltext{distcc} for distributed compilation
    \end{itemize}
  \end{itemize}
\end{frame}
% ------------------------------------------------------------------------------
\begin{frame}
  \frametitle{ccache -- compiler cache}
  \begin{itemize}
      \item{\LARGE{\say{speeds up recompilation by caching previous compilations
        and detecting when the same compilation is being done again}}}
      \item{\url{https://ccache.dev/}}
  \end{itemize}
\end{frame}
% ------------------------------------------------------------------------------
\begin{frame}
  \frametitle{distcc -- distributed compilation for C and C++}
  \begin{itemize}
      \item{\LARGE{\say{distributes compilation of C or C++ code across several
        machines on a network}}}
      \item{\url{https://github.com/distcc/distcc}}
  \end{itemize}
\end{frame}
% ------------------------------------------------------------------------------
\begin{frame}
  \frametitle{\shelltext{make}: uncached clean build vs. 100\% cached}
  \fitfig{time-llvm-8GB-ld-noccache-vs-ccache-1j.pdf}
\end{frame}
% ------------------------------------------------------------------------------
\begin{frame}
  \frametitle{More jobs: \shelltext{make clean \&\& make -j 2}}
  \fitfig{resusage-llvm-16GB-ld-noccache-2j.pdf}
\end{frame}
% ------------------------------------------------------------------------------
\begin{frame}
  \frametitle{That's somewhat better\ldots}
  \begin{itemize}
    \Large
    \item{System resource usage is still low}
    \item{Build takes more than 2 and half hours -- still too long}
    \item{Parallelization shows potential}
  \end{itemize}
\end{frame}
% ------------------------------------------------------------------------------
\begin{frame}
  \frametitle{Keep going: \shelltext{make clean \&\&  make -j 3}}
  \fitfig{resusage-llvm-16GB-ld-noccache-3j.pdf}
\end{frame}
% ------------------------------------------------------------------------------
\begin{frame}
  \frametitle{What happened?}
  \begin{itemize}
    \Large
    \item{System usage starts to climb}
    \begin{itemize}
      \large
      \item{especially memory usage}
      \item{signs of correlation with linker execution}
    \end{itemize}
    \item{Build \emph{failed} after 2 hours 20 minutes -- looooong}
    \begin{itemize}
      \large
      \item{Linux OOM\footnote{Out Of Memory} process killer, kills some linker jobs}
    \end{itemize}
    \item{Parallelization still shows potential}
  \end{itemize}
\end{frame}
% ------------------------------------------------------------------------------
\begin{frame}
  \frametitle{Meh. MOAR jobs! \shelltext{make clean \&\&  make -j 7}}
  \fitfig{resusage-llvm-16GB-ld-noccache-7j.pdf}
\end{frame}
% ------------------------------------------------------------------------------
\begin{frame}
  \frametitle{Not great, not terrible}
  \begin{itemize}
    \Large
    \item{System usage very high toward the end of the build}
    \begin{itemize}
      \large
      \item{especially memory usage}
      \item{still correlated with linker execution}
    \end{itemize}
    \item{Build \emph{failed} after 1 hours 5 minutes -- getting better}
    \begin{itemize}
      \large
      \item{OOM process killer, kills some linker jobs}
    \end{itemize}
    \item{Parallelization rules!}
  \end{itemize}
\end{frame}
% ------------------------------------------------------------------------------
\begin{frame}
  \frametitle{Still come cores left\dots \shelltext{make clean \&\&  make -j 9}}
  \fitfig{reboot-llvm-16GB-ld-noccache-9j.pdf}
\end{frame}
% ------------------------------------------------------------------------------
\begin{frame}
  \frametitle{What have we learned so far}
  \begin{itemize}
    \LARGE
    \item{Most of the build time is spent by compilation}
    \item{Compilation caching -- good.}
    \item{Compilation paralelization -- good.}
    \begin{itemize}
      \Large
    \item{We have not even {\LARGE really used} \shelltext{distcc} yet.}
    \end{itemize}
    \item{Linking paralelization -- bad! Why?}
    \begin{itemize}
      \Large
      \item{Let's have a closer look!}
    \end{itemize}
  \end{itemize}
\end{frame}
% ------------------------------------------------------------------------------
\begin{frame}
  \frametitle{Linker memory usage vs link time -- no cache}
    \fitfig{scatter-llvm-buildtime-ramusage-noccache-8G-8C.pdf}
\end{frame}
% ------------------------------------------------------------------------------
\begin{frame}
  \frametitle{Linker memory usage vs link time -- cached}
    \fitfig{scatter-llvm-buildtime-ramusage-ccache-8G-8C.pdf}
\end{frame}
% ------------------------------------------------------------------------------
\begin{frame}
  \frametitle{Link target memory requirements -- classification}
    \fitfig{hist-llvm-link-mem-req.pdf}
\end{frame}
% ------------------------------------------------------------------------------
\begin{frame}
    \frametitle{Linking \shelltext{llvm} and \shelltext{clang} in a nutshell}
  \begin{itemize}
    \large
    \item{Around 100 different link targets}
    \begin{itemize}
      \normalsize
      \item{Executables, Shared libraries}
      \item{Many of them are {\normalsize small}.}
      \item{Few of the {\LARGE huge}.}
    \end{itemize}
    \item{Most of the linking is done {\Large towards the end} of the build process.}
    \begin{itemize}
      \large
      \item[$\implies$]{Using many parellel jobs, multiple instances of linker run concurrently.}
      \begin{itemize}
        \item[$\implies$]{Many {\Large big targets} are linked {\Large at the same time}.}
        \begin{itemize}
          \item[$\implies$]{\Huge{OOM!}}
        \end{itemize}
      \end{itemize}
    \end{itemize}
    \item{BTW: so much swapping is bad for the SSDs.}
  \end{itemize}
\end{frame}
% ------------------------------------------------------------------------------
\begin{frame}
    \frametitle{Making parallel linking work}
  \begin{itemize}
    \LARGE
    \item Let's try some additional tricks:
    \begin{itemize}
      \Large
      \item Use GNU \shelltext{gold} instead of GNU \shelltext{ld}.
      \begin{itemize}
        \large
        \item[$+$] Faster than \shelltext{ld}.
        \item[$+$] {\Large Uses less memory} than \shelltext{ld}.
	    \normalsize 
		\item[$-$] {Can link only in ELF format.}
      \end{itemize}
      \item Use \shelltext{zram}.
      \begin{itemize}
        \normalsize
        \item a Linux kernel module for creating a compressed block device in RAM
        \item can be used as swap device -- part of swap is in compressed RAM
      \end{itemize}
      \item Make only some targets
      \begin{itemize}
        \normalsize
        \item Practical when {\large developing and debugging}.
        \item Sometimes you just need to do \shelltext{make all}.
      \end{itemize}
    \end{itemize}
  \end{itemize}
\end{frame}
% ------------------------------------------------------------------------------
\begin{frame}
  \frametitle{\shelltext{zram}+\shelltext{distcc}+\shelltext{ccache}+\shelltext{gold}+\shelltext{make -j 12}}
  \fitfig{resusage-llvm-16GB-gold-ccache-12j.pdf}
\end{frame}
% ------------------------------------------------------------------------------
\begin{frame}
    \frametitle{If only\ldots}
  \begin{itemize}
    \LARGE
    \item \ldots we could
    \begin{itemize}
      \Large
      \item prevent so many big targets from being linked at once,
      \item prevent excessive swapping to disk,
    \end{itemize}
    \item but still
    \begin{itemize}
      \Large
      \item use the hardware resources efficiently,
	  \item use available remote cores via \shelltext{distcc} to full extent.
    \end{itemize}
  \end{itemize}
\end{frame}
% - Solution -------------------------------------------------------------------
\section{Solution}
% ------------------------------------------------------------------------------
\begin{frame}
  \frametitle{Let's synchronize the execution of \shelltext{ld}/\shelltext{gold}}
  \begin{itemize}
    \LARGE
    \item \shelltext{atmost} FTW!
    \begin{itemize}
	  \large
      \item Simple utility written in C.
      \item Takes a command-line (executable + arguments).
      \item Waits until specified conditions are met.
	  \item Executes the command-line.
	  \item {\small\url{https://github.com/matus-chochlik/various}/atmost}
	  
    \end{itemize}
  \end{itemize}
\end{frame}
% ------------------------------------------------------------------------------
\begin{frame}[fragile]
  \frametitle{RTFM -- \shelltext{man atmost}}
\scriptsize
\begin{verbatim}
ATMOST(1)                 General Commands Manual                ATMOST(1)

NAME
       atmost  - tool for limiting the concurrent execution of a specified
       executable(s).

SYNOPSIS
       atmost [-v|--verbose] [-f|--file file-path]
              [-s|--socket socket-path] [-r|--reset]
              [-i|--sleep-interval seconds] [-c|--print-current]
              [-C|--print-all-current] [-n|--max-instances count]
              [-l|--max-cpu-load-1m percent]
              [-L|--max-cpu-load-5m percent] [-m|--min-avail-ram percent]
              [-M|--min-free-ram percent] [-S|--min-free-swap percent]
              [-p|--max-total-procs count] [-tc|--max-cpu-temp temp]
              [-tg|--max-gpu-temp temp] [-tb|--max-bat-temp temp]
              [-io|--max-io-ops count] [-nw|--max-nw-speed speed]
              [-snw|--slow-nw-speed speed] [-- executable [args...]]

       atmost --help
\end{verbatim}
\end{frame}
% ------------------------------------------------------------------------------
\begin{frame}
  \frametitle{\shelltext{atmost} -- description}
  \Large
    \say{The atmost utility allows to~{\LARGE limit concurrent execution}
    of~a~single executable
    or~a~set of~executables, depending on~various criteria like CPU load,
    memory and~swap usage, thermal zone temperatures, current number
    of~I/O~operations, etc., or just {\LARGE by a maximum number}.}
\end{frame}
% ------------------------------------------------------------------------------
\begin{frame}
  \frametitle{\shelltext{atmost -n NUMBER -- EXECUTABLE} -- basic usage}
  \begin{itemize}
    \Large
    \item Limits the concurrent execution of \shelltext{EXECUTABLE} to the specified \shelltext{NUMBER} of instances.
    \item Uses an IPC semaphore set\footnote{see \shelltext{man 2 semget}}.
    \begin{itemize}
      \large 
      \item {\em Acquires\footnote{see \shelltext{man 2 semop}}} the semaphore before executing \shelltext{EXECUTABLE}.
      \item {\em Releases} the semaphore after executing \shelltext{EXECUTABLE}.
      \item \shelltext{realpath}\footnote{see \shelltext{man 3 realpath}} of \shelltext{EXECUTABLE} is used as the semaphore set token\footnote{see \shelltext{man 3 ftok}}.
    \end{itemize}
  \end{itemize}
\end{frame}
% ------------------------------------------------------------------------------
\begin{frame}[fragile]
  \frametitle{Using \shelltext{atmost} -- wrapper scripts}
  \begin{itemize}
    \large
    \item In order to synchronize a \shelltext{EXECUTABLE} with \shelltext{atmost}:
    \begin{itemize}
      \item Create a shell script with the name of \shelltext{EXECUTABLE}.
      \item From the script call \shelltext{atmost} with appropriate arguments.
      \item Put the wrapper script to \shelltext{\$\{PATH\}}
    \end{itemize}
  \end{itemize}
  \shelltext{\$ vim /opt/bin/ld}
  \begin{verbatim}
  #!/bin/bash
  atmost -n 2 -- /usr/bin/x86_64-linux-gnu-gold "${@}"
  \end{verbatim}
  \shelltext{\$ chmod u+x /opt/bin/ld}
  \shelltext{\$ export PATH="/opt/bin:\$\{PATH\}"}\\
\end{frame}
% ------------------------------------------------------------------------------
\begin{frame}
  \frametitle{\shelltext{atmost -n 2 -- ld}}
  \begin{itemize}
  \Large
  \item [+] Empirically we have found that 2 instances of \shelltext{ld} can run
    safely with 16GB RAM.
  \item [+] Simple
  \item [+] Low overhead
  \item [-] Too coarse and restrictive
    \begin{itemize}
      \large
      \item The majority of the \shelltext{llvm} targets has low RAM requirements for linking.
      \item Targets from other projects also have low linker RAM requirements.
      \item We could safely use {\Large more parallel link jobs}.
    \end{itemize}
  \end{itemize}
\end{frame}
% ------------------------------------------------------------------------------
\begin{frame}
  \frametitle{What if\dots}
  \begin{itemize}
    \large
  \item \dots we could {\LARGE determine} if it is safe to start the linker {\LARGE per each invocation}?
    \begin{itemize}
      \LARGE
      \item Enter \shelltext{atmost\_driver}!
      \begin{itemize}
        \large
        \item Running as a local \shelltext{AF\_UNIX} socket server.
        \item \shelltext{atmost} is the client (with the \shelltext{--socket PATH} option).
        \item When \shelltext{atmost} is started, it connects to the {\em driver} and sends information about the wrapped \shelltext{EXECUTABLE}.
        \item When the {\em driver} determines that it's safe to run \shelltext{EXECUTABLE}, it responds to \shelltext{atmost}.
        \item Then \shelltext{atmost} executes \shelltext{EXECUTABLE}.
      \end{itemize}
    \end{itemize}
  \end{itemize}
\end{frame}
% ------------------------------------------------------------------------------
\begin{frame}
  \frametitle{Let's make it even more flexible}
  \begin{itemize}
    \LARGE
    \item \shelltext{atmost\_driver} callbacks:
    \begin{itemize}
      \large
      \item The \shelltext{atmost\_driver} implements the reusable {\Large common
        server functionality}.
      \item The specific logic determining when it is safe to start executing,
        is implemented in a {\Large set of callbacks}.
      \item \shelltext{load\_callback\_data}
      \item \shelltext{save\_callback\_data}
      \item \shelltext{process\_initialized}
      \item \shelltext{let\_process\_go}
      \item \shelltext{process\_finished}
    \end{itemize}
  \end{itemize}
\end{frame}
% ------------------------------------------------------------------------------
\begin{frame}
  \frametitle{\shelltext{atmost}+\shelltext{atmost\_driver}+callbacks}
  \fitfig{diagram-components.pdf}
\end{frame}
% ------------------------------------------------------------------------------
\begin{frame}[fragile]
  \frametitle{Callback -- \shelltext{load\_callback\_data}}
  \begin{lstlisting}
  def load_callback_data():
    resources = do_load_resources(...)
    return resources
  \end{lstlisting}
  \begin{itemize}
    \Large
    \item Called once, when the \shelltext{atmost\_driver} is being started.
    \item Typically does startup initialization.
    \item Can load resources used by the other callbacks.
  \end{itemize}
\end{frame}
% ------------------------------------------------------------------------------
\begin{frame}[fragile]
  \frametitle{Callback -- \shelltext{save\_callback\_data}}
  \begin{lstlisting}
  def save_callback_data(resources):
    do_cleanup_resources(resources)
  \end{lstlisting}
  \begin{itemize}
    \Large
    \item Called once, when the \shelltext{atmost\_driver} is being shutdown.
    \item The \shelltext{resource} parameter is the return value from
      \shelltext{load\_callback\_data}.
    \item Can cleanup and/or save resources used and data generated during the run.
  \end{itemize}
\end{frame}
% ------------------------------------------------------------------------------
\begin{frame}[fragile]
  \frametitle{Callback -- \shelltext{process\_initialized}}
  \begin{lstlisting}
  def process_initialized(resources, proc):
    # handle new process
    proc.set_callback_data(callback_data)
  \end{lstlisting}
  \begin{itemize}
    \Large
    \item Called when \shelltext{atmost} sends the driver information about a
      new process.
    \item The \shelltext{proc} parameter provides information about the new process.
    \begin{itemize}
      \normalsize
      \item PID, command-line, environment, working-directory, etc.
      \item The callback can store its bookkeeping data in \shelltext{proc}.
    \end{itemize}
  \end{itemize}
\end{frame}
% ------------------------------------------------------------------------------
\begin{frame}[fragile]
  \frametitle{Callback -- \shelltext{let\_process\_go}}
  \begin{lstlisting}
  def let_process_go(resources, procs):
    if can_process_start(resource, procs):
      return True
    return False
  \end{lstlisting}
  \begin{itemize}
    \Large
    \item Called repeatedly.
    \item Determines if a process can start executing the synchronized executable.
    \item The \shelltext{procs} argument contains info about {\em all} currently
      managed processes, split into 3 groups.
    \begin{itemize}
      \large
      \item \shelltext{active} -- processes that are already executing.
      \item \shelltext{waiting} -- processes that are waiting for execution.
      \item \shelltext{current} -- the process to be let go.
    \end{itemize}
  \end{itemize}
\end{frame}
% ------------------------------------------------------------------------------
\begin{frame}[fragile]
  \frametitle{Callback -- \shelltext{process\_finished}}
  \begin{lstlisting}
  def process_finished(resources, proc):
    callback_data = proc.callback_data()
    # cleanup info about the process
  \end{lstlisting}
  \begin{itemize}
    \Large
    \item Called when a process handled by the driver has finished.
    \item The callback can retrieve its bookkeeping data from \shelltext{proc}.
  \end{itemize}
\end{frame}
% ------------------------------------------------------------------------------
\begin{frame}[fragile]
  \frametitle{RTFM -- \shelltext{man atmost\_driver}}
\scriptsize
\begin{verbatim}
ATMOST(1)                 General Commands Manual                ATMOST(1)

NAME
       atmost_driver - driver server for the atmost command.

SYNOPSIS
       atmost_driver [-s|--socket socket-path] [-c|--callbacks file-path]
                     [-u|--update interval]

       atmost_driver -h
       atmost_driver --help
\end{verbatim}
\end{frame}
% ------------------------------------------------------------------------------
\begin{frame}
  \frametitle{Back to linking\ldots}
  \begin{itemize}
  \Large
  \item What if the memory usage can be determined from the linker command-line
    arguments?
    \begin{itemize}
    \large
    \item[1)] Run a lot of builds of various projects.
    \item[2)] Gather {\Large a lot} of data
      \begin{itemize}
        \item command line arguments,
        \item input file sizes,
        \item actual linker memory usage,
        \item etc.
      \end{itemize}
    \item[3)] {\Large ???}
    \item[4)] {\LARGE Profit!}
    \end{itemize}
  \end{itemize}
\end{frame}
% ------------------------------------------------------------------------------
\begin{frame}
  \frametitle{What exactly is step 3) ???}
  \begin{itemize}
  \Large
  \item Feed the gathered data into a machine learning model and train it.
  \item Use the trained ML model to {\LARGE predict} linker {\LARGE memory usage}
    from command-line arguments.
  \item Integrate the ML classifier into a \shelltext{atmost\_driver} callback
    script.
    \begin{itemize}
      \large
      \item Keep track of available memory.
      \item Predict the memory usage of new instances of linker.
      \item Let the new linkers go only if there is enough available memory.
    \end{itemize}
  \end{itemize}
\end{frame}
% ------------------------------------------------------------------------------
\begin{frame}
  \frametitle{The ML part}
  \begin{itemize}
  \Large
  \item Uses the \shelltext{neural\_network.MLPClassifier} from the
    \shelltext{sklearn}\footnote{\url{https://scikit-learn.org/stable/documentation.html}}
    Python package.
  \item Inputs:
    \begin{itemize}
      \large
      \item optimization level, the PIE\footnote{position-independent executable} flag,
      \item count and combined size of input shared and static object files,
      \item etc.
    \end{itemize}
  \item Output:
    \begin{itemize}
      \large
      \item the predicted memory requirements, as a multiple of 256MB chunks,
      \item i.e. if the output is $5$ the predicted size is $5*256[MB] = 1.25[GB]$.
    \end{itemize}
  \end{itemize}
\end{frame}
% - Results --------------------------------------------------------------------
\section{Results}
% ------------------------------------------------------------------------------
\begin{frame}
  \frametitle{Classifier prediction accuracy}
  \fitfig{prediction-actual-vs-predicted.pdf}
\end{frame}
% ------------------------------------------------------------------------------
\begin{frame}
  \frametitle{Integrating classifier with \shelltext{atmost\_driver}}
  \begin{itemize}
  \Large
  \item \shelltext{load\_callback\_data} -- load the trained ML model.
  \item \shelltext{process\_initialized} -- use the model to predict memory usage.
  \item \shelltext{let\_process\_go} -- only if predicted linker memory usage is
    less than current available memory.
  \item \shelltext{process\_finished} -- output actual memory usage that can be
    stored and used for addtional model training.
  \end{itemize}
\end{frame}
% ------------------------------------------------------------------------------
\begin{frame}
  \frametitle{Linker schedule (no ccache, 8GB RAM) -- \shelltext{make -j 10}}
  \fitfig{gantt-noccache-8G-10j.pdf}
\end{frame}
% ------------------------------------------------------------------------------
\begin{frame}
  \frametitle{Linker schedule (no ccache, 8GB RAM) -- \shelltext{make -j 20}}
  \fitfig{gantt-noccache-8G-20j.pdf}
\end{frame}
% ------------------------------------------------------------------------------
\begin{frame}
  \frametitle{Linker schedule (ccached, 8GB RAM) -- \shelltext{make -j 20}}
  \fitfig{gantt-ccache-8G-20j.pdf}
\end{frame}
% ------------------------------------------------------------------------------
\begin{frame}
  \frametitle{Link order vs. time (no ccache, 8GB, 8+8 cores, 8 jobs)}
  \fitfig{time-llvm-8GB-ld-noccache-1j-8j.pdf}
\end{frame}
% ------------------------------------------------------------------------------
\begin{frame}
  \frametitle{Link order vs. time (no ccache, 8GB, 8+8 cores, 16 jobs)}
  \fitfig{time-llvm-8GB-ld-noccache-1j-16j.pdf}
\end{frame}
% ------------------------------------------------------------------------------
\begin{frame}
  \frametitle{Link order vs. time (ccached, 8GB, 8+8 cores, 16 jobs)}
  \fitfig{time-llvm-8GB-ld-ccache-1j-16j.pdf}
\end{frame}
% ------------------------------------------------------------------------------
\begin{frame}
  \frametitle{Parallelization statistic indicators}
  \begin{itemize}
    \LARGE
  \item {\em Run time {\Large with $j$ jobs}}
    \begin{itemize}
    \item $T_j$
    \end{itemize}
  \item {\em Speedup {\Large with $j$ jobs}}
    \begin{itemize}
    \item $S_j = T_1 / T_j$
    \end{itemize}
  \item {\em Marginal speedup {\Large with $j$ jobs}}
    \begin{itemize}
    \item $M_j = T_{j-1} / T_j$
    \end{itemize}
  \item {\em Efficiency {\Large with $j$ jobs}}
    \begin{itemize}
    \item $E_j = S_j / j$
    \end{itemize}
  \end{itemize}
\end{frame}
% ------------------------------------------------------------------------------
\begin{frame}
  \frametitle{Parallelization statistics (no ccache, 8GB, 8+8 cores)}
  \fitfig{stats-noccache-8G-8C.pdf}
\end{frame}
% ------------------------------------------------------------------------------
\begin{frame}
  \frametitle{Parallelization statistics (ccached, 8GB, 8+8 cores)}
  \fitfig{stats-ccache-8G-8C.pdf}
\end{frame}
% - Conclusion -----------------------------------------------------------------
\section{Conclusion}
% ------------------------------------------------------------------------------
\begin{frame}
  \frametitle{Conclusions}
  \begin{itemize}
    \Large
    \item \shelltext{atmost} is a highly flexible tool for synchronizing
      execution of parallel processes,
    \begin{itemize}
    \large
    \item improves build times by allowing many
      parallel \shelltext{make} jobs while serializing huge link jobs,
    \item prevents system stalls, crashing and reboots due to low memory conditions,
    \item makes building big projects safer,
    \item allows to build \shelltext{llvm} on systems with low RAM capacity,
    \item reduces the amount of writes to swap disk partitions.
    \end{itemize}
  \end{itemize}
\end{frame}
% ------------------------------------------------------------------------------
\begin{frame}
  \frametitle{Further improvements}
  \begin{itemize}
    \Large
    \item Try to improve the prediction of memory usage based on command-line
      arguments.
    \begin{itemize}
      \item Examine the arguments more deeply.
      \item Use a different classifier.
      \item Develop a custom model.
      \item etc.
    \end{itemize}
    \item Allow to change environment variables for the synchronized process
      from the \shelltext{atmost\_driver} callbacks.
  \end{itemize}
\end{frame}
% ------------------------------------------------------------------------------
\begin{frame}
  \centering
  \Huge
  Thank you!\\Questions?
\end{frame}
% - Extras ---------------------------------------------------------------------
\section{Extras}
% ------------------------------------------------------------------------------
\begin{frame}
  \centering
  \Huge
  Extras
\end{frame}
% ------------------------------------------------------------------------------
\begin{frame}
  \frametitle{\shelltext{atmost} -- additional options}
  \begin{itemize}
    \Large
    \item Besides the basic semaphore-based synchronization and the customizable
      driver-based synchronization, \shelltext{atmost} supports synchronization
      depending on:
    \begin{itemize}
      \large
      \item 1 and 5 minutes average load,
      \item amount of available RAM,
      \item amount of free swap,
      \item CPU temperature,
      \item network speed,
      \item battery status,
      \item etc.
    \end{itemize}
  \end{itemize}
\end{frame}
% ------------------------------------------------------------------------------
\begin{frame}[fragile]
  \frametitle{\shelltext{atmost} -- average load}
  \begin{itemize}
    \item \shelltext{-l} or \shelltext{--max-cpu-load-1m}
    \item \shelltext{-L} or \shelltext{--max-cpu-load-5m}
  \end{itemize}

  \large
  Start executing only if 1 minute average load is less than or equal 15\%:
  \normalsize
  \begin{verbatim}
  atmost -l 15 -- executable arguments...
  \end{verbatim}

  \large
  Start executing only if 5 minute average load is less than or equal 8.5\%:
  \normalsize
  \begin{verbatim}
  atmost -L 8.5 -- executable arguments...
  \end{verbatim}
\end{frame}
% ------------------------------------------------------------------------------
\begin{frame}[fragile]
  \frametitle{\shelltext{atmost} -- memory / swap usage}
  \begin{itemize}
    \item \shelltext{-m} or \shelltext{--min-avail-ram}
    \item \shelltext{-M} or \shelltext{--min-free-ram}
    \item \shelltext{-S} or \shelltext{--min-free-swap}
  \end{itemize}

  \large
  Start executing only if at least $20\%$ or RAM is available:
  \normalsize
  \begin{verbatim}
  atmost -m 20 -- executable arguments...
  \end{verbatim}

  \large
  Start executing only if at least $70\%$ or swap is free:
  \normalsize
  \begin{verbatim}
  atmost -S 70 -- executable arguments...
  \end{verbatim}
\end{frame}
% ------------------------------------------------------------------------------
\begin{frame}[fragile]
  \frametitle{\shelltext{atmost} -- total process count}
  \begin{itemize}
    \item \shelltext{-p} or \shelltext{--max-total-procs}
  \end{itemize}

  \large
  Start executing only if the number of currently running processes is $\leq 1000$:
  \normalsize
  \begin{verbatim}
  atmost -p 1000 -- executable arguments...
  \end{verbatim}
\end{frame}
% ------------------------------------------------------------------------------
\begin{frame}[fragile]
  \frametitle{\shelltext{atmost} -- thermal zone temperatures}
  \begin{itemize}
    \item \shelltext{-tc} or \shelltext{--max-cpu-temp}
    \item \shelltext{-tg} or \shelltext{--max-gpu-temp}
    \item \shelltext{-tb} or \shelltext{--max-bat-temp}
  \end{itemize}

  \large
  Start executing only if the CPU temperature is less than or equal
  to $70\degree C$:
  \normalsize
  \begin{verbatim}
  atmost -tc 70 -- executable arguments...
  \end{verbatim}

  \large
  Start executing only if the battery temperature is less than or equal
  to $55.5\degree C$:
  \normalsize
  \begin{verbatim}
  atmost -tb 55.5 -- executable arguments...
  \end{verbatim}
\end{frame}
% ------------------------------------------------------------------------------
\begin{frame}[fragile]
  \frametitle{\shelltext{atmost} -- modifiers}
  \begin{itemize}
    \Large
    \item Modifiers allow to change the limits by a specified amount under
      special conditions.
    \begin{itemize}
      \large
      \item when running on battery,
      \item when connected to \say{slow} network,
      \item when disconnected from network.
    \end{itemize}
  \end{itemize}
\end{frame}
% ------------------------------------------------------------------------------
\begin{frame}[fragile]
  \frametitle{\shelltext{atmost} -- on A/C vs. on battery}
  \begin{itemize}
    \item \shelltext{-batt} or \shelltext{--on-battery}
  \end{itemize}

  \large
  When on A/C start only if 1 minute average load is less than or equal to $15\%$,\\
  when on battery start only if average load is less than or equal to $5\% = (15-10)$:
  \normalsize
  \begin{verbatim}
  atmost -l 15 -batt 10 -- executable arguments...
  \end{verbatim}

\end{frame}
% ------------------------------------------------------------------------------
\begin{frame}[fragile]
  \frametitle{\shelltext{atmost} -- combining limits and modifiers}

  \large
  When on A/C start only if 1 minute load is $\leq 20\%$ and
  5 minutes load is $\leq 10\%$,\\
  when on battery start only if 1 minute load is $\leq 8\%$ and
  5 minutes load is $\leq 4\%$:
  \normalsize
  \begin{verbatim}
  atmost -l 20 -batt 12 -L 10 -batt 6 -- executable ...
  \end{verbatim}

\end{frame}
% ------------------------------------------------------------------------------
\end{document}

