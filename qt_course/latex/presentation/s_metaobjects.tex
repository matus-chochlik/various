\section{The metaobject system}

\begin{frame}
  \frametitle{The metaobject system\footnote
    {\url{http://doc.qt.io/qt-5.6/metaobjects.html}}}
  \begin{itemize}
    \item Simplified run-time reflection mechanism for all types descending
      from \texttt{QObject}.
    \item Provides basic metadata describing classes, their properties,
      member functions, constructors, etc.
    \item Also helps to implement
    \begin{itemize}
      \item the signals and slots mechanism,
      \item the dynamic object properties,
      \item the class-info metadata,
      \item the \texttt{qobject\_cast}.
    \end{itemize}
  \end{itemize}
\end{frame}

\begin{frame}[fragile]
  \frametitle{\texttt{QMetaObject}\footnote
    {\url{http://doc.qt.io/qt-5.6/qmetaobject.html}}}
  \begin{itemize}
    \item Provides meta-information about the actual class of an instance
      of \texttt{QObject}:
      \begin{itemize}
        \item class name,
        \item base class -- \texttt{QMetaClass},
        \item class infos -- \texttt{QMetaClassInfo},
        \item properties -- \texttt{QMetaProperty},
        \item enumerations -- \texttt{QMetaEnum},
        \item methods -- \texttt{QMetaMethod},
        \item constructors -- \texttt{QMetaMethod}.
      \end{itemize}
    \item Obtained by calling \verb@QObject::metaObject@.
  \end{itemize}
\end{frame}

\begin{frame}[fragile]
  \frametitle{\texttt{QMetaClassInfo}\footnote
    {\url{http://doc.qt.io/qt-5.6/qmetaclassinfo.html}}}
  \begin{itemize}
    \item Provides additional constant name and value string pair describing
      arbitrary aspect of the class.
    \item Each \texttt{QObject} can have an arbitrary number of class infos.
    \item Defined by the \verb@Q_CLASSINFO@ macro:
    \begin{verbatim}
	class MyClass
	{
	    Q_OBJECT
	    Q_CLASSINFO("autor", "Name Surname")
	    Q_CLASSINFO("url", "http://example.com")
	    Q_CLASSINFO("email", "name@example.com")
	};
    \end{verbatim}
    \item Obtained by calling \verb@QMetaObject::classInfo@.
  \end{itemize}
\end{frame}

\begin{frame}[fragile]
  \frametitle{\texttt{QMetaProperty}\footnote
    {\url{http://doc.qt.io/qt-5.6/qmetaproperty.html}}}
  \begin{itemize}
    \item Provides metadata describing a single property of a class descending
     from \texttt{QObject}
     \begin{itemize}
       \item \texttt{name}
       \item \texttt{type}
       \item \texttt{typeName}
       \item \texttt{isReadable}
       \item \texttt{isWritable}
       \item \texttt{isResetable}
       \item \texttt{isEnumType}
       \item \texttt{isFinal}
       \item \ldots
     \end{itemize}
    \item Each \texttt{QObject} can have an arbitrary number of properties.
    \item Obtained by calling \verb@QMetaObject::property@.
  \end{itemize}
\end{frame}

\begin{frame}[fragile]
  \frametitle{\texttt{QMetaEnum}\footnote
    {\url{http://doc.qt.io/qt-5.6/qmetaenum.html}}}
  \begin{itemize}
    \item Provides metadata describing an enumeration type defined in a descendant
     of \texttt{QObject}
     \begin{itemize}
       \item \texttt{name}
       \item \texttt{scope}
       \item \texttt{keyCount} -- number of enumeration values
       \item \texttt{value} -- int
       \item \texttt{key} -- string
       \item value to key conversion
       \item key to value conversion
       \item \ldots
     \end{itemize}
    \item Each \texttt{QObject} can have an arbitrary number of enumerations.
    \item Obtained by calling \verb@QMetaObject::enumerator@.
  \end{itemize}
\end{frame}

\begin{frame}[fragile]
  \frametitle{\texttt{QMetaMethod}\footnote
    {\url{http://doc.qt.io/qt-5.6/qmetamethod.html}}}
  \begin{itemize}
    \item Provides metadata describing method defined in a descendant of
      \texttt{QObject}.
      \begin{itemize}
        \item \texttt{name}
        \item \texttt{access} -- private, protected, public
        \item \texttt{parameterCount}
        \item \texttt{parameterNames}
        \item \texttt{parameterTypes}
        \item \texttt{returnType}
        \item \ldots
        \item \texttt{invoke} -- call the method
      \end{itemize}
    \item Obtained by calling \verb@QMetaObject::method@.
  \end{itemize}
\end{frame}

