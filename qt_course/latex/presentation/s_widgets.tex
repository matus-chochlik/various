\subsection{Qt's Widget model}

\begin{frame}
  \frametitle{Qt's widget model -- \texttt{QWidget}\footnote
    {http://doc.qt.io/qt-5.6/qwidget.html}}
  \begin{itemize}
    \item Specialization of \texttt{QObject} for visual components.
    \begin{itemize}
      \item Inherits the memory management functionality, etc.
      \item Adds basic functionality for visual representation.
    \end{itemize}
    \item Base class of graphical user interface components.
    \item Event reception and handling functionality.
    \begin{itemize}
      \item keyboard,
      \item mouse,
      \item window system,
      \item etc.
    \end{itemize}
    \item Responsible for painting the widget on the screen.
    \begin{itemize}
      \item Only draws a gray rectangle.
      \item Overridden in the derived classes.
      \item Can use platform's native components.
      \item Use styles.
    \end{itemize}
  \end{itemize}
\end{frame}

\begin{frame}
  \frametitle{Qt's widget model -- \texttt{QWidget}}
  \begin{itemize}
    \item Similar to \texttt{QObject}, instances of \texttt{QWidget} form an
    N-way tree hierarchy.
    \begin{itemize}
      \item This hierarchy represents widget {\em containment}.
      \item A {\em parent} widget can contain several {\em child} widgets.
      \item Child widget coordinate system is relative to the parent's
        coordinate system.
      \item Child widgets are clipped by parents boundaries.
      \item Most state changes on the parent propagate to the children
      \begin{itemize}
        \item visible / invisible,
        \item enabled / disabled,
        \item etc.
      \end{itemize}
    \end{itemize}
    \item Widgets with no explicitly-assigned parent automatically become desktop windows.
  \end{itemize}
\end{frame}

