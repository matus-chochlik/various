\subsection{Translations and internationalization}

\begin{frame}
  \frametitle{Localization overview\footnote
    {\url{https://wiki.qt.io/QtInternationalization}}}
  \begin{itemize}
    \item Translation of user interface texts to user language
    \begin{itemize}
      \item Left-to-right, right-to-left scripts.
      \item Plural forms.
    \end{itemize}
    \item Displaying information in region-specific format.
    \begin{itemize}
      \item date and time,
      \item numbers and currency,
      \item \ldots
    \end{itemize}
    \item Other considerations
    \begin{itemize}
      \item time-zones,
      \item color sensitivities,
      \item cultural references and habits,
      \item \ldots
    \end{itemize}
  \end{itemize}
\end{frame}

\begin{frame}
  \frametitle{Basic workflow}
  \small
  \begin{itemize}
    \item Mark all text literals which are to be translated in the source.
    \item Update the project files.
    \item Invoke the \texttt{lupdate} tool to generate the \texttt{.ts} files.
    \item Update the \texttt{.ts} translation XML files -- translate the texts
      into a particular language.
    \item Invoke the \texttt{lrelease} tool to generate the \texttt{.qm} binary
      translation files.
    \item Update the application source to load and apply the translations.
    \item Use the \texttt{QLocale} class to convert between numbers and strings
      in locale-specific formats.
  \end{itemize}
\end{frame}

\begin{frame}[fragile]
  \frametitle{Marking texts in source}
  \begin{itemize}
    \item All text literals which should be translated into user's language
     should be wrapped in the \texttt{tr()}\footnote{in C++} or the
      \texttt{qsTr()}\footnote{in QML} macros.
    \item Basic variant
    \begin{itemize}
      \item \verb@tr("&Application")@
      \item \verb@qsTr("&File")@
    \end{itemize}
    \item Disambiguation
    \begin{itemize}
      \item \verb@tr("&Open", "file")@
      \item \verb@tr("&Open", "window")@
      \item \verb@tr("&Open", "network connection")@
    \end{itemize}
    \item Plurals
    \begin{itemize}
      \item \verb@tr("%1 items(s) loaded", "", n)@
      \item \verb@tr("%1 file(s) remaining", "progress", n)@
    \end{itemize}
  \end{itemize}
\end{frame}

\begin{frame}[fragile]
  \frametitle{Updating project files}
  \begin{itemize}
    \item The \texttt{TRANSLATIONS} variable in the project files should
     be updated to contain the names of the translation XML files.
     \begin{lstlisting}
	TRANSLATIONS += myapp_sk.ts
	TRANSLATIONS += myapp_de.ts
	TRANSLATIONS += myapp_fr.ts
     \end{lstlisting}
    \item The \texttt{SOURCES} variable determines which source files should
     be scanned.
    \item It is possible to add sources only for the \texttt{lupdate} tool:
     \begin{lstlisting}
	lupdate_only {
	    SOURCES += gui.qml
	    SOURCES += helper.qml
	}
     \end{lstlisting}
  \end{itemize}
\end{frame}

\begin{frame}[fragile]
  \frametitle{\texttt{lupdate}\footnote
   {\url{http://doc.qt.io/qt-5.6/linguist-manager.html\#lupdate}}}
  \begin{itemize}
    \item The \texttt{lupdate} tool scans the source files and searches for
      marked string literals.
    \item Generates the \texttt{.ts}, XML-based translation files.
    \item Basic usage
    \begin{itemize}
      \item \verb@lupdate project.pro@
      \item \verb@lupdate src.cpp src.qml ... -ts output.ts@
    \end{itemize}
  \end{itemize}
\end{frame}

\begin{frame}[fragile]
  \frametitle{Translating the \texttt{.ts} files}
  \small
  \begin{itemize}
    \item The generated \texttt{.ts} files contain the original strings extracted
      from the source files.
    \item Each \texttt{.ts} file contains the translation of the interface
      to a single language.
    \item These files should be translated by the translators to the target
      language.
  \end{itemize}
  \begin{lstlisting}[basicstyle=\tiny\ttfamily]
<!DOCTYPE TS>
<TS version="2.1">
<context>
    <name>primes</name>
    <message>
        <location filename="primes.qml" line="16"/>
        <source>Quit</source>
        <comment>application</comment>
        <translation>Ukoncit</translation>
    </message>
    <message>
        <location filename="primes.qml" line="23"/>
        <source>Start</source>
        <translation>Spustit</translation>
    </message>
</context>
</TS>
  \end{lstlisting}
\end{frame}

\begin{frame}[fragile]
  \frametitle{\texttt{lrelease}\footnote
   {\url{http://doc.qt.io/qt-5.6/linguist-manager.html\#lrelease}}}
  \begin{itemize}
    \item The \texttt{lrelease} tool converts the XML-based \texttt{.ts} files
      into a more efficient binary representation which is loaded by the
      application.
    \item Basic usage
    \begin{itemize}
      \item \verb@lrelease project.pro@
      \item \verb@lrelease file_sk.ts  ... -qm output.qm@
    \end{itemize}
  \end{itemize}
\end{frame}

\begin{frame}[fragile]
  \frametitle{Loading translation in a QML/C++ application}
  \small
  \begin{itemize}
   \item The binary translation files can be embedded as resources:
   \begin{lstlisting}[basicstyle=\scriptsize\ttfamily]
	<!DOCTYPE RCC><RCC version="1.0">
	<qresource>
	    <file>primes.qml</file>
	    <file>primes_sk.qm</file>
	</qresource>
	</RCC>
   \end{lstlisting}
   \item The \texttt{QTranslator}\footnote
     {\url{http://doc.qt.io/qt-5.6/qtranslator.html}} class can be used to load
     a translation and register it with the application
   \begin{lstlisting}[basicstyle=\scriptsize\ttfamily]
	QGuiApplication app(argc, argv);

	QTranslator translator;
	if(translator.load(":/primes_sk.qm")) {
	    app.installTranslator(&translator);
	}
   \end{lstlisting}
  \end{itemize}
\end{frame}

\begin{frame}[fragile]
  \frametitle{\texttt{QLocale}\footnote
   {\url{http://doc.qt.io/qt-5.6/qlocale.html}}}
  \begin{itemize}
    \item Converts data like numbers, date and time, currency, etc. between
      their numeric value and string representation in a specified language.
   \item Enumerations
   \begin{itemize}
     \small
     \item \texttt{QLocale::Language}
     \item \texttt{QLocale::Script}
     \item \texttt{QLocale::Country}
     \item \texttt{QLocale::MeasurementSystem}
     \item \ldots
   \end{itemize}
   \item Construction:
   \begin{itemize}
     \small
     \item \verb@QLocale::system()@ -- system locale
     \item \verb@QLocale(const QString& name)@ -- \verb@sk_SK@, \verb@en_US@,
       \verb@de_DE@, etc.
     \item \verb@QLocale(Language, Country)@
     \item \verb@QLocale(Language, Script, Country)@
   \end{itemize}
  \end{itemize}
\end{frame}

\begin{frame}
  \frametitle{\texttt{QLocale} -- functions}
  \small
  \begin{itemize}
    \item \texttt{country}, \texttt{language}, \texttt{script}, \texttt{name}
    \item \texttt{nativeCountryName}, \texttt{nativeLanguageName}
    \item \texttt{textDirection},
    \item \texttt{dateFormat}, \texttt{timeFormat}, \texttt{dateTimeFormat}
    \item \texttt{currencySymbol}, \texttt{percent}, \texttt{zeroDigit},
      \texttt{exponential}
    \item \texttt{positiveSign}, \texttt{negativeSign}, \texttt{decimalPoint},
      \texttt{groupSeparator}
    \item \texttt{toString}, \texttt{toCurrencyString}
    \item \texttt{toInt}, \texttt{toFloat}, \texttt{toDouble},
      \texttt{toDate}, \texttt{toTime}, etc.
    \item \texttt{toUpper}, \texttt{toLower}
      
  \end{itemize}
\end{frame}

