\section{Event-driven programming}

\begin{frame}
  \frametitle{Event-driven programming}
  \footnotesize
  \begin{itemize}
    \item The flow of the program is determined by {\em events} -- objects
     representing various changes in the state of:
     \begin{itemize}
       \item user input devices -- mouse, keyboard, touchscreen, sensors, etc.
       \item file system -- file created, deleted, renamed, etc.
       \item input / output devices -- data available on serial port, socket, etc.
       \item the window system -- widget position, size, Z-order or focus,
         screen-saver activation, hibernation, reboot, etc.
       \item power management -- battery fully charged or almost drained, etc.
       \item timers -- timeout elapsed
       \item application lifetime -- initializing, initialized, closing, etc.
       \item \ldots
     \end{itemize}
    \item Objects representing events are generated by various subsystems and
      pushed into an {\em event queue}.
    \item The main loop of the program repeatedly pops events from the event queue
      and dispatches them to appropriate {\em receivers}, {\em listeners} or
      {\em handlers}.
  \end{itemize}
\end{frame}

\section{Events}

\begin{frame}
  \frametitle{Qt Events\footnote
    {\url{http://doc.qt.io/qt-5.6/eventsandfilters.html}}}
  \small
  \begin{itemize}
    \item In Qt events are represented by instances with types derived from
     \texttt{QEvent}.
    \item Events can be received and handled by any instance of \texttt{QObject}
      but are mostly relevant to \texttt{QWidget}s.
    \item When a logical event happens its object representation is created by the
      appropriate subsystem.
    \item Qt then delivers the event instance to the {\em receiving}
      \texttt{QObject} by calling its virtual \texttt{QObject::event} function.
    \item Classes derived from \texttt{QObject}, most notably various widgets,
      implement their own specialized virtual functions handling only particular
      event types, which can be overriden in widget subclasses.
      \begin{itemize}
        \item \texttt{QWidget::closeEvent} -- handles \texttt{QCloseEvent}s.
        \item \ldots
      \end{itemize}
    \item Handling events requires sub-classing -- not convenient.
  \end{itemize}
\end{frame}

\begin{frame}
  \frametitle{\texttt{QEvent}\footnote
    {\url{http://doc.qt.io/qt-5.6/qevent.html}}}
  \small
  \begin{itemize}
    \item Common base class for specialized event types
    \item Subclasses represent different event types -- containing values
      relevant to the particular event types.
    \begin{columns}
      \tiny
      \begin{column}{0.32\textwidth}
      \begin{itemize}
        \item \texttt{QActionEvent}
        \item \texttt{QChildEvent}
        \item \texttt{QCloseEvent}
        \item \texttt{QDropEvent}
        \item \texttt{QEnterEvent}
        \item \texttt{QExposeEvent}
        \item \texttt{QFileOpenEvent}
        \item \texttt{QFocusEvent}
        \item \texttt{QGestureEvent}
        \item \texttt{QHelpEvent}
        \item \texttt{QHideEvent}
      \end{itemize}
      \end{column}
      \begin{column}{0.36\textwidth}
      \begin{itemize}
        \item \texttt{QIconDragEvent}
        \item \texttt{QGraphicsSceneEvent}
        \item \texttt{QDragLeaveEvent}
        \item \texttt{QDynamicPropertyChangeEvent}
        \item \texttt{QStateMachine::SignalEvent}
        \item \texttt{QStateMachine::WrappedEvent}
        \item \texttt{QWindowStateChangeEvent}
        \item \texttt{QInputMethodQueryEvent}
        \item \texttt{QPlatformSurfaceEvent}
        \item \texttt{QWhatsThisClickedEvent}
        \item \texttt{QInputMethodEvent}
      \end{itemize}
      \end{column}
      \begin{column}{0.32\textwidth}
      \begin{itemize}
        \item \texttt{QInputEvent}
        \item \texttt{QMoveEvent}
        \item \texttt{QPaintEvent}
        \item \texttt{QResizeEvent}
        \item \texttt{QScrollEvent}
        \item \texttt{QScrollPrepareEvent}
        \item \texttt{QShortcutEvent}
        \item \texttt{QShowEvent}
        \item \texttt{QStatusTipEvent}
        \item \texttt{QTimerEvent}
      \end{itemize}
      \end{column}
    \end{columns}
  \end{itemize}
\end{frame}

