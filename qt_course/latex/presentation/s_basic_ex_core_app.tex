\begin{frame}[fragile]
  \frametitle{Basic examples -- \texttt{QCoreApplication}}
  Located in \texttt{examples/02\_basic/06\_core\_app/}.\\
  File \texttt{fib\_calc.hpp}:
  \begin{lstlisting}
	#include <QtCore>

	class MyFibCalc
 	 : public QObject
	{
	private:
	    Q_OBJECT
	    int n;
	    static int _dumbFib(int k);
	public:
	    MyFibCalc(QObject* parent, int i);
	public slots:
	    void calculate(void);
	signals:
	    void done(void);
	};
  \end{lstlisting}
\end{frame}

\begin{frame}[fragile]
  File \texttt{fib\_calc.cpp}:
  \begin{lstlisting}
	MyFibCalc::MyFibCalc(QObject* parent, int i)
 	 : QObject(parent)
 	 , n(i)
	{ }

	int MyFibCalc::_dumbFib(int k)
	{
	    if(k < 2) return 1;
	    return _dumbFib(k-2) + _dumbFib(k-1);
	}

	void MyFibCalc::calculate(void)
	{
	    qDebug() << "fib(" << n << ") = " << _dumbFib(n);
	    emit done();
	}
  \end{lstlisting}
\end{frame}

\begin{frame}[fragile]
  File \texttt{main.cpp}:
  \begin{lstlisting}
	int main(int argc, char *argv[])
	{
	    QCoreApplication app(argc, argv);

	    MyFibCalc* calc = new MyFibCalc(&app(*@\footnotemark@*), 43);

	    QObject::connect(
	        calc, SIGNAL(done()),
	        &app, SLOT(quit())
	    );

	    QTimer::singleShot(0, calc, SLOT(calculate()));

	    return app.exec();
	}
  \end{lstlisting}
  \footnotetext{\texttt{app} becomes parent and owner of \texttt{calc}}
\end{frame}

\begin{frame}[fragile]
  \frametitle{Basic examples -- \texttt{QCommandLineParser}}
  Located in \texttt{examples/02\_basic/07\_clarg\_parser/}.\\
  \small
  Extends the previous example with a command line parser.\\
  Fragment of file \texttt{main.hpp}:
  \begin{lstlisting}[basicstyle=\tiny\ttfamily]
	QCoreApplication app(argc, argv);
	QCoreApplication::setApplicationName("MyFib");
	QCoreApplication::setApplicationVersion("1.1");
	
	QCommandLineParser clap;
	clap.setApplicationDescription("Calculates fibonacci numbers, slowly!");
	clap.addHelpOption();
	clap.addVersionOption();

	const int def_n = 42;

	QCommandLineOption numopt(QStringList() << "n", "number", "Number", "N");
	numopt.setDefaultValue(QString::number(def_n, 10));
	clap.addOption(numopt);
	clap.process(app);

	bool n_ok = true;
	int n = clap.value(numopt).toInt(&n_ok);
  \end{lstlisting}
\end{frame}

\begin{frame}[fragile]
  \frametitle{Basic examples -- \texttt{QSettings}}
  Located in \texttt{examples/02\_basic/08\_settings/}.\\
  \small
  Extends the previous example with persistent settings.\\
  Fragment of file \texttt{main.hpp}:
  \begin{lstlisting}[basicstyle=\tiny\ttfamily]
	QCoreApplication app(argc, argv);
	QCoreApplication::setOrganizationName("FRI");
	// ...
	QCommandLineParser clap;
	// ...

	bool n_ok = true;
	int n = clap.value(numopt).toInt(&n_ok);
	if(!n_ok)
	{
	    QSettings settings;
	    n = settings.value("parameters/n", def_n).toInt(&n_ok);
	}
	else
	{
	    QSettings settings;
	    settings.beginGroup("parameters");
	    settings.setValue("n", n);
	    settings.endGroup();
	}
  \end{lstlisting}
\end{frame}

