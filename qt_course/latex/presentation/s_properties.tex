\section{Properties}

\begin{frame}
  \frametitle{Object properties\footnote
    {\url{http://doc.qt.io/qt-5.6/properties.html}}}
  \begin{itemize}
    \item Qt implements a compiler-independent object property system.
    \item Properties superficially look like class data members, but implement
      additional features:
      \begin{itemize}
        \item \textbf{\em read} accessor function returning the value of the property.
        \item {\em write} accessor function invoked when the property value is set.
        \item {\em reset} function, resetting the property into a default state.
        \item {\em notify} signal, invoked when the property value is changed.
      \end{itemize}
    \item Properties depend on the metaobject system and part of their code
      is generated by \texttt{moc}.
    \item Properties can be manipulated through the GUI in Qt Creator.
    \item Properties can also be manipulated through a generic interface defined
      in \texttt{QObject} by their names.
  \end{itemize}
\end{frame}

\begin{frame}[fragile]
  \frametitle{Declaring properties}
  \footnotesize
  \begin{columns}
    \begin{column}{0.5\textwidth}
    \begin{itemize}
      \item Properties are declared by the \texttt{Q\_PROPERTY} macro:
      \begin{verbatim}
	Q_PROPERTY(type name
	           READ getFunction
	           [WRITE setFunction]
	           [RESET resetFunction]
	           [NOTIFY notifySignal]
	           [REVISION int]
	           [DESIGNABLE bool]
	           [SCRIPTABLE bool]
	           [STORED bool]
	           [USER bool]
	           [CONSTANT]
	           [FINAL])
      \end{verbatim}
    \end{itemize}
    \end{column}
    \begin{column}{0.5\textwidth}
    \begin{itemize}
      \item Read-only property
      \begin{verbatim}
	Q_PROPERTY(bool active
	           READ isActive)
      \end{verbatim}
      \item Read-write property
      \begin{verbatim}
	Q_PROPERTY(bool visible
	           READ isVisible
	           WRITE setVisible)
      \end{verbatim}
      \item Read-write-reset property
      \begin{verbatim}
	Q_PROPERTY(QCursor cursor
	           READ getCursor
	           WRITE setCursor
	           RESET resetCursor)
      \end{verbatim}
    \end{itemize}
    \end{column}
  \end{columns}
\end{frame}

\begin{frame}
  \frametitle{\texttt{Q\_PROPERTY} -- parameters}
  \footnotesize
  \begin{itemize}
    \item \texttt{READ} -- the name of the getter function. Either this or the
      \texttt{MEMBER} parameter must be specified, all other parameters are
      optional.
    \item \texttt{MEMBER} -- the name of a data member storing the value of
      the property. Either this or the \texttt{READ} parameter must be specified.
    \item \texttt{WRITE} -- the name of the setter function.
    \item \texttt{RESET} -- the name of the reset function.
    \item \texttt{NOTIFY} -- the name of an existing signal in the same class.
    \item \texttt{DESIGNABLE} -- whether the property should be visible to the
      form designer. True by default.
    \item \texttt{SCRIPTABLE} -- whether the property should be accessible to the
      scripting engine. True by default.
    \item \texttt{STORED} -- whether the property must be stored when saving the
      object's state.
    \item \texttt{USER} -- whether the property is the primary and usually only
      one to be manipulated by the GUI user\footnote{opposed to the designer}.
    \item \texttt{CONSTANT} -- whether the property is constant.
  \end{itemize}
\end{frame}

