\subsection{Layouts}

\begin{frame}
  \frametitle{Qt's widget layouts -- \texttt{QLayout}\footnote
    {\url{http://doc.qt.io/qt-5.6/qlayout.html}}}
  \begin{itemize}
    \item Layouts are classes responsible for automatic arranging of
    child widgets within a parent widget, in order to make good
    use of the available screen space.
    \item Layouts manage the following tasks:
    \begin{itemize}
      \item positioning of child widgets,
      \item determining sensible default sizes of child widgets,
      \item determining sensible minimum (and maximum) dimensions of child widgets,
      \item resizing of children after the resize of their parent widget,
      \item automatic repositioning when the contents of a parent widget change.
    \end{itemize}
  \end{itemize}
\end{frame}

\begin{frame}
  \frametitle{Qt's widget layouts}
  \begin{itemize}
    \item Similar to widgets, layouts can also form a hierarchy of:
    \begin{itemize}
      \item \texttt{QWidget}s,
      \item other layouts,
      \item frames, margins and other \say{empty} spaces, both fixed and stretching.
    \end{itemize}
    \item Layouts are \texttt{QObject}s, however they {\em are not} descendants
     of \texttt{QWidget} themselves.
    \item Layouts {\em do not} manage the lifetime of the objects they are
      arranging.
    \item Layouts can be applied to instances of \texttt{QWidget}.
    \item The widget to which a layout is applied is responsible for its deletion.
  \end{itemize}
\end{frame}

\begin{frame}
  \frametitle{The layout process}
  \begin{itemize}
    \item The process of creating widget layouts usually consists of these
      steps:
    \begin{itemize}
      \item Create the parent widget
      \item Create the child widgets as being contained and owned by the parent
        widget
      \item Create the layout(s) arranging the children of the parent widget
      \item Add the child widgets to the layout(s)
      \item Optionally nest the layout(s)
      \item Assign the single outermost layout to the parent widget
    \end{itemize}
  \end{itemize}
\end{frame}

\begin{frame}
  \frametitle{Basic layout types\footnote{Source \url{http://doc.qt.io/qt-5.6/layout.html}}}
  This table briefly describes the most commonly used layout types.
  \begin{center}
  \rowcolors{2}{green!80!yellow!50}{green!70!yellow!40}
  \begin{tabular}{|p{0.2\textwidth}|p{0.7\textwidth}|}
    \hline
    \textbf{Class} & \textbf{Description} \\
    \hline
    \texttt{QHBoxLayout} & Arranges widgets in a single horizontal row, from left
      to right, or right to left for right-to-left languages. \\
    \hline
    \texttt{QVBoxLayout} & Arranges widgets in a single vertical column, from top
      to bottom. \\
    \hline
    \texttt{QGridLayout} & Arranges widgets in a two-dimensional grid with several
      rows and/or columns. Widgets can occupy multiple cells. \\
    \hline
    \texttt{QFormLayout} & Arranges widgets in a 2-column descriptive
      label-and-field style, typical for data input forms. \\
    \hline
  \end{tabular}
  \end{center}
\end{frame}

\begin{frame}[fragile]
  \frametitle{Applying layouts to widgets}
  \begin{itemize}
    \item Layouts are applied to instances of \texttt{QWidget}:
    \begin{itemize}
    \item {\em early} -- by passing a pointer to the parent of the layout in
    the layout's constructor:
    \begin{verbatim}
	QWidget* parent = ...
	QHBoxLayout* layout = new QHBoxLayout(parent);
	QWidget* child = ...
	layout->addWidget(child);
    \end{verbatim}
    \item {\em later} -- by calling \verb@QWidget::setLayout@:
    \begin{verbatim}
	QWidget* parent = ...
	QHBoxLayout* layout = new QHBoxLayout;
	QWidget* child = ...
	layout->addWidget(child);
	parent->setLayout(layout);
    \end{verbatim}
    \end{itemize}
  \end{itemize}
\end{frame}

